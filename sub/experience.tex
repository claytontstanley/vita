%----------------------------------------------------------------------------------------
%	Experience SECTION
%----------------------------------------------------------------------------------------

\begin{rSection}{Experience}

  \begin{rSubsection}{Rice University, Computer Human Interaction Laboratory}
    {Aug 2012 - Present}{Graduate Research Assistant}{Houston, TX}
  \item Research in human-based tagging of online content.
    Built a cognitively inspired Bayesian tag classifier that predicts user-generated tags on StackOverflow posts at 65\% accuracy.
  \end{rSubsection}

  \begin{rSubsection}{Rice University, Computer Human Interaction Laboratory}
    {May 2012 - Aug 2012}{Research Programmer}{Houston, TX}
  \item Migrated 50K lines of Macintosh Common Lisp (MCL) GUI code to Clozure Common Lisp (CCL).
    Implemented subset of the MCL GUI specification in CCL, so that CCL could run original MCL code.
  \end{rSubsection}

  \begin{rSubsection}{Air Force Research Laboratory, Cognitive Models and Agents}
    {May 2009 - May 2012}{Cognitive Scientist and Software Engineer}{Dayton, OH}
  \item
    Ensured seamless transition of network connectivity, software, \& services.
    Successfully designed and implemented a transition path for the lab relocation from Mesa, AZ to Wright-Patterson AFB. 
  \item 
    Kickstarted a core shift in AFRL IT policy priority: from restricting, to empowering the researcher.
    Led first organization at WPAFB to certify and connect to the Defense Research Engineering Network.
  \item 
    Enabling Teraflops of free computing power for the AF.
    Developer for the net-centric MindModeling volunteer computing research project.
    Part of core dev team that redesigned and reimplemented the entire system between 2010-2011.
    Systems-level project. 10,000+ SLOC. 10+ programming languages. 
  \end{rSubsection}

  \begin{rSubsection}{Rice University, Computer Human Interaction Laboratory}
    {May 2007 - May 2009}{Graduate Research Assistant}{Houston, TX}
  \item Explored the mechanisms involved in producing various phenomena found in human visual search tasks
    (e.g., distractor ratio effect, global effect) through modeling with the ACT-R cognitive architecture. 
  \end{rSubsection}

  \begin{rSubsection}{United States Air Force Academy, Department of Physics}
    {May 2006 - May 2007}{Undergraduate Research Assistant}{Colorado Springs, CO}
  \item Researched, redesigned, and reimplemented a parallelized lightcurve inversion program developed in Matlab.
    Successfully ported and run on Hoku (Cray XD-1 Linux system), located at MHPCC.
  \end{rSubsection}

%------------------------------------------------

\end{rSection}
