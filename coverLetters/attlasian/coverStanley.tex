%\documentclass{resume} % Use the custom resume.cls style
\documentclass{article}
\usepackage{parskip} % So that all paragraphs are class ``noindent'' in the generated .html file
\usepackage[none]{hyphenat} % No hypens at line breaks anywhere
\raggedright % Only left justified

\begin{document}

I am interested in mining large datasets to better understand, model, and predict the core relationships in those datasets.
I specialize in modeling human behavioral data, but I also have experience modeling physics-based systems,
so I've applied the same core analytic and mathematical approach to modeling systems across a broad range of domains.

I would like to work at the intersection of data science and user experience research.
I am finishing my doctorate within the Computer Human Interaction Laboratory at Rice University.
So I have been exposed to many of the traditional Human Factors / HCI / User Experience research techniques, and I am passionate about building more intuitive and usable human-machine systems.

I also have almost a decade of HPC experience, so I know how to work with large datasets, optimize code,
and accelerate the research pace by speeding up the turnaround time to finish the various steps in the data analysis process. 
From 2012-2014 my dissertation work is focused on evaluating and improving Bayesian and vector-based models of declarative memory on large-scale datasets.
This work has required many data mining and data science techniques: local, single-core optimization using a high performance workstation, R, data.table, logistic regression, PostgreSQL, and others.
From 2009-2012 I was part of the small development team that completely redesigned and rewrote the MindModeling volunteer distributed HPC system (uses BOINC and technology similar to Folding@HOME).
That web-based system is currently being used by cognitive scientists within the department of defense to better explore the entire performance space of their cognitive models.

I passionate about using analytics techniques to help better understand the data, to improve overall user experience of end products and help the organization make more informed strategic decisions.
But I am also passionate about building the proper infrastructure and tools so that the process of exploring the data can be accelerated.
It's great when data can be used to improve human-machine systems and help guide decision making.
But it's even better when those improvements can be discovered and implemented at a rapid pace.

Thank you for considering me for the position,

-Clayton Stanley

\end{document}
