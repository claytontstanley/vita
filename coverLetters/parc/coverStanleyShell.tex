%\documentclass{resume} % Use the custom resume.cls style
\documentclass{article}
\usepackage{parskip} % So that all paragraphs are class ``noindent'' in the generated .html file
\usepackage[none]{hyphenat} % No hypens at line breaks anywhere
\raggedright % Only left justified

\begin{document}

I am interested in working for PARC to model user behavior for several reasons:
[1] With my computational cognitive modeling, mathematics, and data mining background I have the necessary skills to excel in the position,
[2] I'm actually using the PMI equation developed at PARC for the Bayesian co-occurrence model in my dissertation (The ACT-R declarative memory context model is the same thing once the dataset is large),
[3] and I have about 8 years of experience enjoying programming in Lisp, and the job posting mentioned Lisp as a desired language, and that's rare.
Although I'm enjoying not using Common Lisp for analysis and modeling at the moment (I'm using the more common R and Python currently), if you have a specific research topic that Common Lisp is well suited for,
I'm interested and probably qualified for that as well.

As far as my general research background, I am interested in analyzing large datasets to better understand, model, and predict the core relationships in those datasets.
I specialize in modeling human behavioral data, but I also have experience modeling physics-based systems,
so I've applied the same core analytic and mathematical approach to modeling systems across a broad range of domains.

I also have almost a decade of HPC experience, so I know how to optimize code, and accelerate the research pace by speeding up the turnaround time to finish the various steps in the data analysis process. 
From 2009-2012 I was part of the small development team that completely redesigned and rewrote the MindModeling volunteer distributed HPC system (uses BOINC and technology similar to Folding@HOME).
That web-based system is currently being used by cognitive scientists within the department of defense to better explore the entire performance space of their cognitive models.

From 2012-2014 my dissertation work is focused on evaluating and improving models of declarative memory on large-scale datasets.
I am using these models of long term memory to predict the tags and hashtags that authors use on StackOverflow and Twitter when creating content.
This work has required a lot of data mining and data science techniques: proper tokenization of unstructured text, local optimization using a single high-performance workstation, R, data.table, PostgreSQL, and others.

I passionate about using data to build models that are customized to each user's needs.
These models can help improve user experience, optimize organizational resources, and ensure that the organization is making the best strategic decisions.
But I am also passionate about building the proper infrastructure and tools so that the process of exploring and modeling the data can be accelerated.
It's great when data can be used to build more intuitive, responsive, and usable systems, and to help the organization make better decisions.
But it's even better when those improvements can be discovered and implemented at a rapid pace.

Thank you for considering me for the position,
-Clayton Stanley

\end{document}
