%\documentclass{resume} % Use the custom resume.cls style
\documentclass{article}
\usepackage{parskip} % So that all paragraphs are class ``noindent'' in the generated .html file
\usepackage[none]{hyphenat} % No hypens at line breaks anywhere
\raggedright % Only left justified

\begin{document}

To the MD Anderson biostatistics department,

I have almost a decade of HPC experience using various technologies and designs.
In 2006, I implemented a physics-based model that could be used to identify small orbiting debris using that object's reflective signature of the sun.
That model was written in Matlab, and successfully ported and run on a Linux cluster-based HPC system.
From 2009-2012 I was part of the small development team that completely redesigned and rewrote the MindModeling volunteer distributed HPC system (uses BOINC and technology similar to Folding@HOME).
That web-based system was written in various front-end and back-end technologies, and is currently being used by cognitive scientists to better explore the entire performance space of their cognitive models.
From 2012-2014 my dissertation work is focused on evaluating and improving models of declarative memory on large-scale datasets.
This work has required a lot of data-mining and data-science techniques: local, single-core optimization using a high-performance workstation, R, data.table, PostgreSQL, and others.
So I have a wide range of HPC experience, and I've been exposed to a broad range of environments, techniques, and workflows that are used to validate, explore, and visualize computational models.

With my Physics and Psychology background, I have a mix of formal training in mathematics, physics, cognitive science, statistics, and experimental design.
Most of my statistics and experimental design experience has come from Psychology, but due to my Physics background am also comfortable using and deriving closed-form nonlinear models.

My mix of experience may or may not be what you are looking for.
If you want someone that has a primary focus on only experimental design, or only statistics, or the modeling aspect, or visualization, or the implementation side of things, or just workflow optimization,
I'm probably not the right candidate.
But if you want someone that has broad experience in all of those aspects, and is comfortable and interested working at the intersection of all of these areas of focus, then I'm eager to contribute to the work.  

Thank you for your consideration,

-Clayton Stanley

\end{document}
