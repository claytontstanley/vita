%\documentclass{resume} % Use the custom resume.cls style
\documentclass{article}
\usepackage{parskip} % So that all paragraphs are class ``noindent'' in the generated .html file
\usepackage[none]{hyphenat} % No hypens at line breaks anywhere
\raggedright % Only left justified

\begin{document}

I am interested in analyzing large datasets to better understand, model, and predict the core relationships in those datasets.
I specialize in building predictive models of human behavior, but I also have experience modeling physics-based systems,
so I've applied the same core analytic and mathematical approach to modeling systems across a broad range of domains.

I specialize in computational and mathematical modeling, and I also have a degree in Physics, so I am comfortable using nonlinear statistical modeling techniques to better explain relationships in the data when necessary.
I also have almost a decade of HPC experience, so I know how to work with distributed systems, large datasets, optimize code,
and accelerate the research pace by speeding up the turnaround time to finish the various steps in the data mining process. 
From 2009-2012 I was part of the small development team that completely redesigned and rewrote the MindModeling volunteer distributed HPC system (uses BOINC and technology similar to Folding@HOME).
That web-based system is currently being used by cognitive scientists within the department of defense to better explore the entire performance space of their cognitive models.

From 2012-2014 my dissertation work is focused on evaluating and improving models of declarative memory on large-scale datasets composed of hundreds of millions to billions of data points.
I am exploring these memory models and using them to predict tag and hashtag use on StackOverflow and Twitter.
This work has required a lot of data mining and data-science techniques: local, multi-core optimization using a high-performance workstation, R, data.table, PostgreSQL, and others.

I passionate about using data to build models that are customized and responsive to the specific environments that they are situated in.
These models can help optimize organizational resources, improve workflows, and ensure that the organization is making the best strategic decisions.
But I am also passionate about building the proper infrastructure and tools so that the process of exploring and modeling the data can be accelerated.
It's great when data can be used to build more intuitive, responsive, and usable systems, and to help the organization make better decisions.
But it's even better when those improvements can be discovered and implemented at a rapid pace.

Thank you for considering me for the Data Scientist position at Lockheed Martin,

-Clayton Stanley

\end{document}
