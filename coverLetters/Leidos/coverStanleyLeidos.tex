%\documentclass{resume} % Use the custom resume.cls style
\documentclass{article}
\usepackage{parskip} % So that all paragraphs are class ``noindent'' in the generated .html file
\usepackage[none]{hyphenat} % No hypens at line breaks anywhere
\raggedright % Only left justified

\begin{document}

I have almost a decade of HPC experience using various technologies and designs.
In 2006, I implemented a physics-based model that could be used to identify small orbiting debris using that object's reflective signature of the sun.
That model was written in Matlab, and successfully ported and run on a Linux cluster-based system at MHPCC.
From 2009-2012 I was part of the small development team that completely redesigned and rewrote the MindModeling volunteer distributed HPC system (uses BOINC and technology similar to Folding@HOME).
That web-based system is currently being used by cognitive scientists to better explore the entire performance space of their cognitive models.
From 2012-2014 my dissertation work is focused on evaluating and improving models of declarative memory on large-scale datasets.
This work has required a lot of data mining and data-science techniques: local, single-core optimization using a high-performance workstation, R, data.table, PostgreSQL, and others.
So I have a wide range of HPC experience, and I've been exposed to a broad range of environments, techniques, and workflows that are used to validate, explore, and visualize computational models.

With my Physics and Psychology background, I have a mix of formal training in mathematics, physics, cognitive science, statistics, and experimental design.
I am interested in applying that experience to improve the user experience for scientists and engineers that use HPCs to explore their computational models.
This could include (as examples): Establishing and improving a web-based submission system for a more user-friendly job submission and management process,
better data visualization tools, code optimization / vectorization, porting, improved source control management workflows, etc.
In short, making sure that scientists and engineers can be as productive as possible when using HPC equipment, technologies, and methodologies.

Thank you for your consideration,

-Clayton Stanley

\end{document}
