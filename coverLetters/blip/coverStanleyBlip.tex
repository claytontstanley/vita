%\documentclass{resume} % Use the custom resume.cls style
\documentclass{article}
\usepackage{parskip} % So that all paragraphs are class ``noindent'' in the generated .html file
\usepackage[none]{hyphenat} % No hypens at line breaks anywhere
\raggedright % Only left justified

\begin{document}

I have almost a decade of computational modeling experience using various technologies and designs.
In 2006, I implemented a physics-based model that could be used to identify small orbiting debris using that object's reflective signature of the sun.
That model was written in Matlab, and successfully ported and run on a Linux cluster-based system at MHPCC.
From 2009-2012 I was part of the small development team that completely redesigned and rewrote the MindModeling volunteer distributed HPC system (uses BOINC and technology similar to Folding@HOME).
That web-based system is currently being used by cognitive scientists to better explore the entire performance space of their cognitive models.
From 2012-2014 my dissertation work is focused on evaluating and improving Bayesian and random permutation statistical models of long-term memory on large-scale datasets.
This work has required a lot of data mining and data-science techniques: local, single-core optimization using a high performance workstation, R, data.table, PostgreSQL, and others.
So I have a wide range of analysis, modeling and development experience, and I've been exposed to a broad range of environments, techniques, and workflows that are used to validate, explore, and visualize computational models.

With my Physics and Psychology background, I have a mix of formal training in mathematics, physics, cognitive science, statistics, and experimental design.
I am interested in applying that experience to discover, understand, and then leverage important statistical patterns in behavioral data from large-scale datasets.

Although I do not have the typical CS PhD that I'm figuring you are looking for, I see CS as the means to the end, but not the end in itself.
CS provides the tools for data scientists to be productive, but their tasks are often applied to understanding human behavioral data.
So the relationships embedded in the data are based on human behavior, and a trained quantitative cognitive modeler has a solid set of domain expertise to understand those relationships.
And my programming aptitude should speak for itself from my past job experience and work.
So I have both the tools to be a productive data scientist, along with the domain expertise to better understand the human behavioral data that data scientists often study.

I'm working on my Portuguese.
I'm a basic level speaker and writer and intermediate reader.
I'm hooked on the language, have visited Brazil several times but I haven't yet had the chance to visit Portugal.

Thanks for considering my application,
-Clayton Stanley

\end{document}
