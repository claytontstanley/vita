%\documentclass{resume} % Use the custom resume.cls style
\documentclass{article}
\usepackage{parskip} % So that all paragraphs are class ``noindent'' in the generated .html file
\usepackage[none]{hyphenat} % No hypens at line breaks anywhere
\raggedright % Only left justified

\begin{document}

I would like to work at the intersection of data science and user experience research.
I am finishing my doctorate within the Computer Human Interaction Laboratory at Rice University.
So I have been exposed to many of the traditional Human Factors / HCI / User Experience research techniques, and I am passionate about building more intuitive and usable human-machine systems.

I specialize in computational and mathematical modeling, and I also have a degree in Physics, so I am comfortable using nonlinear statistical modeling techniques to better explain relationships in the data when necessary.
I also have almost a decade of HPC experience, so I know how to work with large datasets, optimize code,
and accelerate the research pace by speeding up the turnaround time to finish the various steps in the data mining process. 
From 2012-2014 my dissertation work is focused on evaluating and improving models of declarative memory on large-scale datasets composed of hundreds of millions to billions of data points.
I am exploring these memory models by applying them to the task of predicting tag and hashtag use on StackOverflow and Twitter.
This work has required a lot of data mining and data-science techniques: local, single-core optimization using a high-performance workstation, R, data.table, PostgreSQL, and others.
From 2009-2012 I was part of the small development team that completely redesigned and rewrote the MindModeling volunteer distributed HPC system (uses BOINC and technology similar to Folding@HOME).
That web-based system is currently being used by cognitive scientists within the department of defense to better explore the entire performance space of their cognitive models.

I passionate about using data mining, modeling, and statistical techniques to help better understand, model, and improve the overall user experience. 
But I am also passionate about building the proper infrastructure and tools so that the process of exploring the data can be accelerated.
It's great when data can be used to build more intuitive, responsive, and usable systems.
But it's even better when those improvements can be discovered and implemented at a rapid pace.
As a UX Data Scientist at Bloomberg, I will help mine and model user data to better understand and improve the user experience.
And I will also help improve the infrastructure for going through this process so that the research pace for the whole team is accelerated.

Thank you for considering me for the position,

-Clayton Stanley

\end{document}
