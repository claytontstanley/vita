%\documentclass{resume} % Use the custom resume.cls style
\documentclass{article}
\usepackage{parskip} % So that all paragraphs are class ``noindent'' in the generated .html file
\usepackage[none]{hyphenat} % No hypens at line breaks anywhere
\raggedright % Only left justified

\begin{document}

I'm interested in growing and applying my HCI and data science expertise as a quantitative user experience researcher.
I'd like to work on projects that are at the intersection of user experience research and large-scale behavioral datasets.
That is, I'd like to mine and explore the large datasets of user behavior that are becoming more and more common, discover statistical patterns of user behavior in those datasets, extract that information,
and apply it towards improving the human-computer system.

My dissertation is focusing on how we can leverage large datasets of human created content from Twitter and StackOverflow to improve our understanding of human declarative memory.
As the size of these human-created datasets has grown,
it has become possible to evaluate psychological theories of declarative memory on large scale real-world tasks where the theory can be tested on hundreds of millions to billions of data points.
I will evaluate how well current declarative memory theories can model the tags/hashtags that users choose to use when composing a tweet on Twitter or a post on StackOverflow.
Working with these large scale datasets provides a rich testbed to explore modifications to current theories of declarative memory.
It also enables researchers to begin to test these theories at a scale that is approaching the number of memory elements stored in the human declarative memory system.

From an applied HCI perspective, the dissertation work aims to produce accurate tag recommendation models for real-world tagging tasks.
These models could help guide the user to generate more relevant tags for their content, which in turn helps connect the user to the content (and community) that they care about more quickly.

If accepted to a Google user experience research internship program this summer,
I'd like to focus on the applied HCI perspective and look for other ways that the large behavioral datasets that Google has amassed (e.g., mail, youtube)
can be mined and explored to discover patterns of user behavior, gain a better understanding of a user's goals, develop models of the statistical patterns, 
and use the results to improve the human-machine system to be more responsive, customized, and more in tuned with each user.

\end{document}
