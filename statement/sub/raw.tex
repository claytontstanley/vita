My research interests are centered on developing smart, responsive, and highly usable human-computer systems.
I approach this problem by using mathematical and statistical techniques to model human behavior for a specific human-machine system, and then leverage that model to improve the system in some way.
I am mostly interested in domains in which large sets of unstructured user-performance data for a particular task have been amassed.
Data mining statistical techniques and tools can be used on these large unstructured datasets to discover statistical patterns of human behavior in the data.
Embedded within these patterns are crucial pieces of information concerning how the user currently interacts with a particular system.

For example, datasets for online social media sites might contain frequency information on how favorable (or unfavorable) a newly deployed interface feature is,
or which feature requests the users want the most.
Datasets for online mail services might contain statistical patterns in how users write calendar dates in emails.
These patterns can then be leveraged to create a user feature in which calendar dates in emails informing of upcoming events are turned into hyperlinks.
Datasets for a deployed human-software system may also generate massive streams of timing and event information concerning the precise sequence of interactions that a user performs with the system.
These data streams contain patterns of common human errors with the system, and can be mined to discover the most common mistakes people are making when interacting with the system.
Discovering these common errors is a crucial step to figuring out how to improve the human-computer system so that error rates are minimized.

Taken a step further, statistical patterns in these large-scale human behavioral datasets can also be modeled.
These models provide a cleaner representation of how the key variables of interest in the data interact.
They also enable building predictive systems that can more accurately anticipate future user behavior, and can help identify a user's goals and interests.
For example, these models can be used to build smart recommendation systems by mining prior user performance with a particular system.
These recommendation systems can be applied in wide-spanning domains: from marketing and advertising to spam detection to recommending hashtags on Twitter when users are composing a tweet.

I specialize in this modeling aspect of the data analysis process.
I prefer to use models that are both cognitively plausible and have shown to be robust, useful, and relevant to the more general AI machine-learning fields.
Models that lie at the intersection of cognitive plausibility and high performance.
Leveraging models from the general machine-learning AI community ensures that the models both scale and have high prediction.
Leveraging models from the Psychology community ensures that the prediction patterns of the resulting model best follow the behavioral data.
