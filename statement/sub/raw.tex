My research interests are centered on developing smart, responsive, and highly usable human-computer interfaces.
I approach this HCI problem by using mathematical and statistical techniques to model human behavior for a specific interface, and then leverage that model to improve the interface in some way.
I specialize in ``Big Data'' domains where large sets of unstructured user-performance data for a particular task have been amassed.
Data mining statistical techniques and tools can be used on these large unstructured datasets to discover statistical patterns of human behavior in the data.
Embedded within these patterns are crucial pieces of information concerning how the user currently interacts with and feels about a particular computer interface.

For example, datasets for online social media sites might contain frequency information on how favorable (or unfavorable) a newly deployed interface feature is,
or which feature requests the users want the most.
Datasets for online mail services might contain statistical patterns in how users write calendar dates in emails.
These patterns can then be leveraged to create a user feature where calendar dates in emails informing of upcoming events are turned into hyperlinks.
A user can then click on these links to automatically add that event to their calendar.
Datasets for a deployed human-software system may also generate massive streams of timing and event information concerning the precise sequence of interactions that a user performs with the system.
These data streams contain patterns of common human errors with the system, and can be mined to discover the most common mistakes people are making when interacting with the system.
Discovering these common errors is a crucial step to figuring out how to improve the human-computer interface so that error rates are minimized.

I prefer using cognitively-plausible statistical and mathematical models when developing theories that can adequately describe and predict human behavior for a particular task.
The basic hypothesis is that since humans are performing the task, the best way to describe their behavior is to use models that have similar constraints that humans have when performing the task.
I have experience with the ACT-R cognitive architecture, have developed cognitive behavioral models with that architecture,
but have also used mathematical pieces of the architecture to build statistical models.
One of the core components of ACT-R is the declarative memory system.
It turns out that the constraints of human performance when retrieving knowledge from long-term memory are best described by a Bayesian statistical model.
And further, the Bayesian model that best fits the human data describes a retrieval system that is performing optimally (in statistical terms) at the task.
And these same Bayesian models are commonly used in the AI machine-learning fields, since they have strong performance there.

Holographic reduced representations (HRRs) are another type of model that both performs well when tested on general machine-learning tasks
and is a cognitively-plausible account of long-term human memory retrieval.

Both Bayesian and HRR models can be used to build smart recommendation systems by mining prior user performance with a particular system.
These recommendation systems can be applied in wide-spanning domains: from marketing and advertising to recommending hashtags on Twitter when users are composing a tweet.

So these are the types of models that I prefer to use:
Models that are both cognitively plausible and have shown to be robust, useful, and relevant to the more general AI machine-learning fields.
Models that lie at the intersection of cognitive plausibility and high performance.
Leveraging models from the general machine-learning AI community ensures that the models both scale and have high prediction.
Leveraging models from the Psychology community ensures that the prediction patterns of the resulting model best follow the behavioral data.
But further, these psychological models also computationally scale, since humans are constrained information processors.
