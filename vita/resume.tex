%%%%%%%%%%%%%%%%%%%%%%%%%%%%%%%%%%%%%%%%%
% Medium Length Professional CV
% LaTeX Template
%
% This template has been downloaded from:
% http://www.LaTeXTemplates.com
%
% Original author:
% Trey Hunner (http://www.treyhunner.com/)
%
% Important note:
% This template requires the resume.cls file to be in the same directory as the
% .tex file. The resume.cls file provides the resume style used for structuring the
% document.
%
%%%%%%%%%%%%%%%%%%%%%%%%%%%%%%%%%%%%%%%%%

%----------------------------------------------------------------------------------------
%	PACKAGES AND OTHER DOCUMENT CONFIGURATIONS
%----------------------------------------------------------------------------------------

\documentclass{resume} % Use the custom resume.cls style

% Biblatex package and dependencies
\usepackage[american]{babel}
%\usepackage[utf8]{inputenc}


\usepackage{csquotes}
\usepackage[left=0.75in,top=0.6in,right=0.75in,bottom=0.6in]{geometry} % Document margins
\usepackage[style=apa,sortcites=true,sorting=noneyear,backend=biber]{biblatex}
\DeclareLanguageMapping{american}{american-apa}
\addbibresource{bibliography.bib}

\DeclareSortingScheme{noneyear}{
	\sort[direction=descending]{\field{year}}
}

\name{John Smith} % Your name
\address{123 Broadway \\ City, State 12345} % Your address
\address{(000)~$\cdot$~111~$\cdot$~1111 \\ john@smith.com} % Your phone number and email

\begin{document}


\nocite{*}

%----------------------------------------------------------------------------------------
%	EDUCATION SECTION
%----------------------------------------------------------------------------------------

\begin{rSection}{Education}
{\bf Rice University} \hfill {\em May 2009} \\ 
M.A. in Psychology with emphasis in computational cognitive modeling \\
Thesis title: Visual Displays: Developing a Computational Model Explaining the Global Effect \\
Advisor: Dr. Mike Byrne \\
Overall GPA: FIXME 

{\bf United States Air Force Academy} \hfill {\em May 2007} \\ 
B.S. in Physics with emphasis in applied physics and computational methods \\
Distinguished Graduate \\
Overall GPA: FIXME
\end{rSection}

%----------------------------------------------------------------------------------------
%	Awards SECTION
%----------------------------------------------------------------------------------------

\begin{rSection}{Awards}
Awarded United States Air Force Commendation Medal \hfill {\em May 2012}
\item Received 'Kenneth R. Laughery Award', best master’s thesis in Psychology, Rice University \hfill {\em May 2009}
\item Awarded 'Outstanding Cadet in Applied Physics', United States Air Force Academy \hfill {\em May 2007}
\item Won 'Cadet Inter-Service Computer Programming Competition', USAFA \hfill {\em May 2007}
\end{rSection}

%----------------------------------------------------------------------------------------
%	Research Interests SECTION
%----------------------------------------------------------------------------------------

\begin{rSection}{Research Interests}
Human Computer Interaction, Analysis \& Modeling of Large-Scale Human Behavioral Datasets
\end{rSection}

%----------------------------------------------------------------------------------------
%	Experience SECTION
%----------------------------------------------------------------------------------------

\begin{rSection}{Experience}

\begin{rSubsection}{Rice University, Computer Human Interaction Laboratory}{May 2012 - Aug 2012}{Research Programmer}{Houston, TX}
\item Migrated lab's Macintosh Common Lisp GUI code to Clozure Common Lisp. 
\item Implemented the necessary subset of Digitool's MCL GUI specification, so all MCL GUI code could be compiled by Clozure Common Lisp without any code modifications.
\item Heavy use of Clozure's Objective-C / Cocoa bridge, CLOS and extending Clozure's easygui class, multiple inheritance / mixins, read- and compile-time macros.
\end{rSubsection}

\begin{rSubsection}{Air Force Research Laboratory, Cognitive Models and Agents}{May 2009 - May 2012}{Software Engineer and Cognitive Scientist}{Dayton, OH}
\item Developer for the net-centric MindModeling volunteer computing research project.
Part of small (3 person) development team that redesigned and reimplemented the entire system between 2010-2011.
Systems-level project. 10,000+ SLOC. 10+ programming languages.
\item Built a generic wrapper in Common Lisp that allows modelers to easily
(within minutes) interface their model in a form that is compatible with High Performance Computing and Distributed Computing clusters.
Codebase is used on MindModeling system.
Successfully transitioned the code to the rest of the team before departure.
\item Built a tool in Bash and GNU Make that automatically hardens \& certifies OS X and OpenSUSE machines to be used on the Air Force research network.
Deployed to a production environment, and other labs are now using and contributing to the codebase.
\end{rSubsection}

\begin{rSubsection}{Rice University, Computer Human Interaction Laboratory}{May 2007 - May 2009}{Graduate Student}{Houston, TX}
\item Explored the underlying mechanisms involved in producing various phenomena found in visual search tasks
(e.g., distractor ratio effect, global effect) through modeling empirical data using the ACT-R cognitive architecture. 
\end{rSubsection}

\begin{rSubsection}{United States Air Force Academy, Department of Physics}{May 2006 - May 2007}{Research Assistent}{Colorado Springs, CO}
\item Helped research, restructure, and reprogram a parallelized lightcurve inversion program developed in Matlab.
Successfully ported and run on Hoku (Cray XD-1 Linux computing system), located at MHPCC.
\end{rSubsection}

%------------------------------------------------

\end{rSection}

%----------------------------------------------------------------------------------------
%	TECHNICAL STRENGTHS SECTION
%----------------------------------------------------------------------------------------

\begin{rSection}{Technical Strengths}

\begin{tabular}{ @{} >{\bfseries}l @{\hspace{6ex}} l }
Computer Languages & 		Common Lisp, GNU Make, Bash, Regex, R, Matlab, PHP, \\
& 				SQL, Python \\
Programming Paradigms &		Macros, Anaphoric/Read/Compile Macros, DSL programming, \\
& 				object oriented, functional, HOFs, closures, declarative, \\
&				code parallelization \& optimization \\
Protocols \& APIs & 		ACT-R, Cocoa, BOINC, Clozure's Objective C bridge \\
Databases &			PostgreSQL, MySQL \\
Networking \& Infrastructure &	VMWare virtualization technologies, SAN configuration, \\
&				Network Services: Apache, SQL, VNC, sshd, Samba \\
Tools & 			git, Vim, command-line, High-Performance Computers, \\
&				Jenkins, \LaTeX, FogBugz \\
Team Processes & 		Agile, Scrum, daily standups, restrospectives, code reviews, \\
& 				bug tracking, test-driven development \\
Relevant Coursework &		Statistics: univariate, multivariate, psychometrics, regression \\
&				Mathematics: e.g., partial differential equations, discrete math
\end{tabular}

\end{rSection}


%----------------------------------------------------------------------------------------
%	Membership & Service SECTION
%----------------------------------------------------------------------------------------

\begin{rSection}{Membership \& Service}
US Air Force Active-Duty Veteran, Captain (Sep) \hfill {\em 30 May 2007 - 31 May 2012}
\end{rSection}

%----------------------------------------------------------------------------------------
%	Publications SECTION
%----------------------------------------------------------------------------------------

\begin{rSection}{Publications}

\begin{rSubsection}{Peer-Reviewed Journals}{}{}{}
\printbibliography[keyword=journal,heading=none]
\end{rSubsection}

\begin{rSubsection}{Conference Proceedings}{}{}{}
\printbibliography[keyword=proceeding,heading=none]
\end{rSubsection}

\end{rSection}

%\begin{rSection}{Section Name}

%Section content\ldots

%\end{rSection}

%----------------------------------------------------------------------------------------

\end{document}
